\documentclass[16pt]{article}
\usepackage[russian]{babel}
\usepackage{a4wide}
\usepackage[utf8]{inputenc}
\usepackage{graphicx}
\usepackage{amsmath}
\usepackage{amssymb}
\usepackage{amsthm}
\usepackage{import}
\usepackage{xifthen}
\usepackage{pdfpages}
\usepackage{transparent}

\newcommand{\incfig}[2]{%
    \def\svgwidth{#2 mm}
    \import{./figures/}{#1.pdf_tex}
}

\newtheorem{Th}{Теорема}
\newenvironment{Proof}{\par\noindent{\bf Доказательство.}}{\hfill$\scriptstyle\blacksquare$}
\newenvironment{Sol}{\par\noindent{\it Решение:}}


\DeclareMathOperator{\Arccos}{arccos}
\DeclareMathOperator{\Arctg}{arctg}
\DeclareMathOperator{\Cl}{Cl}
\DeclareMathOperator{\SI}{SI}
\DeclareMathOperator*{\Argmax}{Argmax}
\DeclareMathOperator*{\Var}{Var}


\newcommand\Real{\mathbb{R}} 
\newcommand\A{(\cdot)} 
\newcommand\Sup[2]{\rho( #1 \, | \, #2 )}
\newcommand\Sum[2]{\sum\limits_{#1}^{#2}}
\newcommand\Scal[2]{\langle #1,\, #2 \rangle}
\newcommand\Norm[1]{\left\| #1 \right\|}
\newcommand\Int[2]{\int\limits_{#1}^{#2}}
\newcommand\PS{\mathcal{P}}
\newcommand\X{\mathcal{X}} 
\newcommand\Pict[3]{
\begin{figure}[h!]
    \centering
    \incfig{#1}{#3}
    \caption{#2}
    \label{fig:#1}
\end{figure}
}



\begin{document}
\paragraph{Домашнее задание от 14-ого дня карантина}
\paragraph{Задача 14.10.(б, в, г)} Найти сопряженный к опервтору $A\colon l_1 \to l_1.$
\begin{itemize}
\item[(б)] $Ax = (\lambda_1 x_1, \lambda_2 x_2, \ldots ), \quad |\lambda_n| \leqslant 1, \ n \in \mathbb{N}$;
\item[(в)] $Ax = (0, x_1, x_2, \ldots)$;
\item[(г)] $Ax= (x_2, x_3, \ldots)$.
\end{itemize} 

\begin{Sol}
Поскольку $l_1^* = l_\infty$, сопряженный оператор $A^*\colon l_\infty \to l_\infty$.
\begin{itemize}
\item[(б)] $\forall y \in l_\infty \ \forall x \in l_1 \quad (A^*y)(x) = y(Ax) = \sum_{n=1}^\infty 
(\lambda_n y_n)x_n$, откуда получаем, что\\ $$A^*y = (\lambda_1 y_1, \lambda_2 y_2, \ldots ).$$
\item[(в)] $\forall y \in l_\infty \ \forall x \in l_1 \quad (A^*y)(x) = y(Ax) = y_2 x_1 + y_3 x_2 + \ldots$,
 поэтому $$A^*y = (y_2, y_3, \ldots).$$
\item[(г)] Аналогично прошлому пункту показывается, что $$A^*y = (0, y_1, y_2, \ldots).$$
\end{itemize} 
\end{Sol}

\paragraph{Задача 14.18} Оператор $A\colon l_2 \to l_1, \ Ax = x$ определен на множестве \\
$$D(A) = \left\{x \in l_2 \colon \sum\limits_{n=1}^{\infty}|x_n| < \infty\right\}.$$
\begin{itemize}
\item[(а)] Доказать, что $ \overline{D(A)} = l_2$.
\item[(б)] Доказать, что $A$ не является ограниченным на $D(A)$.
\item[(в)] Найти $A^*$ и $D(A^*)$.
\end{itemize}

\begin{Sol}
\begin{itemize}
\item[(а)] $D(A)$ является всюду плотным в $l_2$, поскольку любой элемент $x = (x_1, x_2, \ldots)
 \in l_2$ можно приблизить последовательностью
$$x^{\left(n\right)} = (x_1, \ldots, x_n, 0, 0, \dots) \in D(A),$$ при этом\\
$$\|x^{\left(n\right)} - x\|^2 = 
\sum\limits_{k = n+1}^\infty x_k^2 \to 0.$$
\item[(б)] Рассмотрим последовательность $$x_n = \frac{\sqrt{6}}{\pi}\left(1, \frac{1}{2}, \ldots, \frac{1}{n}, 0, \ldots\right) \in D(A)$$.
Норма ее элементов $\|x_n\|_{l_2} \leqslant 1$, однако $\|Ax\|_{l_1} = \|x_n\|_{l_1} \to \infty$, откуда следует,
 что функционал $A$ не является ограниченным.
\item[(в)] Оператор $A^* \colon l_\infty \to l_2$ существует и определен однозначно. Это следует из пункта (a) и теоремы 1 параграфа 18.4 \cite{tren}. При этом $$\forall y \in l_\infty \ \forall x \in l_2 \quad
(A^*y)(x) = y(Ax) = \sum_{n=1}^\infty y_n x_n,$$ поэтому $A^*y =y$, и, поскольку $A^*y \in l_2$, оператор определен
на множестве $$D(A^*) = \left\{y \in l_\infty \colon \sum_{n=1}^\infty y_n^2 < \infty \right\}.$$
\end{itemize}
\end{Sol}

\paragraph{Задача 14.19} Оператор $A\colon L_2[0, 1] \to L_2[0, 1], \ (Ax)(t) = x(t^2)$ определен на множестве \\
$$D(A) = \left\{x\A \in L_2[0,1] \colon \Int{0}{1}x^2(t^2) < \infty\right\}.$$
\begin{itemize}
\item[(а)] Доказать, что $\overline{D(A)} = L_2[0,1]$.
\item[(б)] Доказать, что $A$ не является ограниченным на $D(A)$.
\item[(в)] Найти $A^*$ и $D(A^*)$.
\end{itemize}

\begin{Sol}
\begin{itemize}
\item[(а)] Следует из того, что тригонометрическая система лежит в $D(A)$ и является плотным в $L_2[0,1]$
 множеством.
\item[(б)] Рассмотрим последовательность
$$
x_n(t) =
\begin{cases}
0, & t \in [0, \frac{1}{n}) \\
\frac{1}{\sqrt{2}}t^{-\frac{1}{4}}, & t \in [\frac{1}{n}, 1].
\end{cases}
$$
Заметим, что $$\Norm{x_n\A}^2 = \frac12 \Int{\frac{1}{n}}{1}\frac{1}{\sqrt{t}}dt = 1 - \frac{1}{\sqrt{n}} \to 1,$$
 однако
$$\Norm{Ax_n}^2 = \frac12 \Int{\frac{1}{n}}{1}\frac{1}{t}dt = \frac12 \ln n \to \infty,$$
поэтому оператор не является ограниченным.

\item[(в)] Так как пространство $L_2[0, 1]$ гильбертово, верно соотношение
$$ \Scal{y}{Ax} = \Scal{x}{A^*y},$$
при этом $\Scal{y}{Ax} = \Int{0}{1}y(t)x(t^2)dt = \{s = t^2\} = \Int{0}{1}x(s)\frac{y(\sqrt{s})}{2\sqrt{s}}ds,$ откуда
$$ (A^*y)(t) = \frac{y(\sqrt{t})}{2\sqrt{t}}.$$
Поскольку споряженный оператор принимает значения из $L_2[0, 1]$, его областью определения является множество
$$D(A^*) = \left\{y\A \in L_2[0,1] \colon \Int{0}{1} \frac{y^2(\sqrt{t})}{t}dt < \infty\right\}.$$
\end{itemize}
\end{Sol}

\newpage
\paragraph{Задача 7} Найти сопряженный оператор и его область определения к
оператору\\ $A\colon L_2[0, 1] \to L_2[0, 1], \ (Ax)(t) = x'(t)$, определенному на множестве \\
$$D(A) = \{x\A \in L_2[0,1] \colon x\A \text{\ непрерывно дифференцируема}, \ x(0) = x(1) = 0\}.$$

\begin{Sol}
Множество $\{\sin(\pi n x) ,\ n \in \mathbb{Z}\} \subset D(A)$ является плотным в $L_2[0,1]$ множеством, поэтому 
$\overline{D(A)} = L_2[0, 1]$, и сопряженный оператор определен однозначно. 

$$\Scal{Ax}{y} = \Int{0}{1} x'(t)y(t)dt = x(t)y(t)\Big|_0^1 - \Int{0}{1} x(t)y'(t)dt = - \Int{0}{1} x(t)y'(t)dt
=\Scal{x}{A^*y}.$$

Таким образом, $$(A^*y)(t) = -y'(t).$$

Однако интегрирование по частям  возможно, если $y\A$ непрерывно дифференцируема. Из теоремы 2 п.18.4. \cite{tren} следует замкнутость $D(A^*)$, поэтому область определения является пополнением пространства непрерывно дифференцируемых функций, то есть пространством Соболева $H^1[0, 1]$,
и $y'\A$ рассматривается как обобщенная производная в смысле Соболева. Интегрирование по частям для функций из 
$y\A \in H^1[0,1]$ можно получить предельным переходом, приближая $y\A$ непрерывно дифференцируемыми функциями. 
\end{Sol}

\newpage
\begin{thebibliography}{0}
\bibitem{tren}
	Треногин~В.А. Функциональный анализ, --- М.: Наука, 1980.
\end{thebibliography} 
\end{document}

