\documentclass[16pt]{article}
\usepackage[russian]{babel}
\usepackage{a4wide}
\usepackage[utf8]{inputenc}
\usepackage{graphicx}
\usepackage{amsmath}
\usepackage{mathrsfs}  
\usepackage{amssymb}
\usepackage{amsthm}
\usepackage{import}
\usepackage{xifthen}
\usepackage{pdfpages}
\usepackage{transparent}

\newcommand{\incfig}[2]{%
    \def\svgwidth{#2 mm}
    \import{./figures/}{#1.pdf_tex}
}

\newtheorem{Th}{Теорема}
\newenvironment{Proof}{\par\noindent{\bf Доказательство.}}{\hfill$\scriptstyle\blacksquare$}
\newenvironment{Sol}{\par\noindent{\it Решение:}}


\DeclareMathOperator{\Arccos}{arccos}
\DeclareMathOperator{\Arctg}{arctg}
\DeclareMathOperator{\Cl}{Cl}
\DeclareMathOperator{\Ker}{ker}
\DeclareMathOperator{\Dim}{dim}
\DeclareMathOperator{\Ima}{im}
\DeclareMathOperator*{\Argmax}{Argmax}
\DeclareMathOperator*{\Var}{Var}


\newcommand\Real{\mathbb{R}} 
\newcommand\A{(\cdot)} 
\newcommand\Sup[2]{\rho( #1 \, | \, #2 )}
\newcommand\Sum[2]{\sum\limits_{#1}^{#2}}
\newcommand\Scal[2]{\langle #1,\, #2 \rangle}
\newcommand\Norm[1]{\left\| #1 \right\|}
\newcommand\Int[2]{\int\limits_{#1}^{#2}}
\newcommand\PS{\mathcal{P}}
\newcommand\X{\mathcal{X}} 
\newcommand\Pict[3]{
\begin{figure}[h!]
    \centering
    \incfig{#1}{#3}
    \caption{#2}
    \label{fig:#1}
\end{figure}
}



\begin{document}
\paragraph{Домашнее задание от 56-ого дня карантина (the last but not the least)}
\paragraph{Задача 20.3.} Показать, что оператор $A\colon L_2[0,\,1] \to L_2[0,\,1], $
$$(Ax)(s) = \Int{0}{1}st(1-st)x(t)dt$$
является вполне непрерывным и найти его спектр.

\begin{Sol}
Запишем оператор в более приятном виде
$$(Ax)(s) = s\Int{0}{1}tx(t)dt - s^2\Int{0}{1}t^2x(t)dt.$$
Образ единичного шара равномерно ограничен, так как
$$\Norm{Ax}^2 \leqslant \{(a + b)^2 \leqslant 2(a^2 + b^2)\} \leqslant
2\Int{0}{1}s^2ds\left(\Int{0}{1}tx(t)dt\right)^2 + 2\Int{0}{1}s^4ds\left(\Int{0}{1}t^2x(t)dt\right)^2 \leqslant M,$$
так как все интегралы здесь равномерно ограничены.

$$\Norm{(Ax)(\cdot + h) - (Ax)\A}^2 \leqslant h\Norm{\Int{0}{1}tx(t)dt} + h\Norm{(2s+h)\Int{0}{1}t^2x(t)dt}$$
Обе нормы здесь равномерно ограничены, и потому $\Norm{(Ax)(\cdot + h) - (Ax)\A} \to 0$ равномерно по $x\A$.

Из критерия предкомпактности в $L_2$ следует полная непрерывность $A$.

Найдем теперь собственные значения $A$. При $\lambda = 0$ собственным подпространством является ортогональное
дополнение к $\mathscr{L}\{t, \,t^2\}$.

Для нахождения ненулевых собственных значений запишем уравненение $Ax = \lambda x$:
$$s\Int{0}{1}tx(t)dt - s^2\Int{0}{1}t^2x(t)dt = \lambda x(s)$$
Домножая на $s$, интегрируя и чуть-чуть преобразуя, получим
$$\left(\frac13 - \lambda\right)\Int{0}{1}tx(t)dt = \frac{1}{4}\Int{0}{1}t^2 x(t)dt$$
Домножая на $s^2$ первоначальное выражение, интегрируя и еще чуть-чуть преобразуя,
$$\frac14 \Int{0}{1}tx(t)dt = \left(\frac15 + \lambda\right)\Int{0}{1}t^2x(t)dt$$
Деля одно уравнение на другое ($ \Int{0}{1}tx(t)dt \not= 0$, иначе $\lambda = 0$) и в последний раз
чуть-чуть преобразуя, имеем
$$16(1 - 3\lambda)(1 + 5\lambda) = 15$$
Корни этого квадратного уравнения будут являться собственными значениями оператора $A$.
\end{Sol} 

\newpage
\paragraph{Задача 20.5.} Пусть $A$ --- самосопряженный оператор. Доказать, что собственные векторы, соответствующие
различным собственным, ортогональны.

\begin{Sol}
Пусть $\lambda_1 \not= \lambda_2$, и $x_1$ и $x_2$ --- соответствующие собственные векторы. Тогда
$$\lambda_1\Scal{x_1}{x_2} = \Scal{Ax_1}{x_2} = \Scal{x_1}{Ax_2} = \lambda_2\Scal{x_1}{x_2},$$
откуда $\Scal{x_1}{x_2} = 0$, что и требовалось доказать.
\end{Sol}

\paragraph{Задача 20.11.} Пусть $A$ --- самосопряженный оператор в $\mathbb{H}$, причем 
$\Ima(A-\lambda I) = \mathbb{H}$. Доказать, что $\lambda \in \rho(A)$.

\begin{Sol}
$A - \lambda I$ также является самосопряженным (как сумма двух самосопряженных), поэтому
$\Ker (A-\lambda I) = \{0\}$. Но это означает, что оператор $A -\lambda I$ обратим, и, по теореме
Банаха об обратном операторе, непрерывно обратим, откуда $\lambda \in \rho(A)$.
\end{Sol}

\paragraph{Задача 20.12.} $A$ --- самосопряженный оператор, $\lambda \in \Real, \ \lambda \in \rho(A)$. Доказать,
что резольвента $R_\lambda(A)$ является самосопряженым оператором.

\begin{Sol}
Рассмотрим произвольные $x,\ y \in \mathbb{H}$.
Так как $B = A - \lambda I$ непрерывно обратим, $\Ima B = \mathbb{H}$, и $x = Bz$.
$$\Scal{R_\lambda x}{y} = \Scal{R_\lambda Bz}{y} = \Scal{z}{y},$$
$$\Scal{x}{R_\lambda y} = \Scal{Bz}{R_\lambda y} = \Scal{z}{BR_\lambda y} = \Scal{z}{y},$$
откуда и следует самосопряженность оператора $R_\lambda$.
\end{Sol}

\paragraph{Задача 20.15.} Пусть $A$ --- вполне непрерывный самосопряженный оператор, $\mathbb{H}$ --- бесконечномерное
гильбертово пространство. Пусть оператор $A$ имеет конечное множество собственных значений 
$\lambda_1, \ldots, \lambda_n$. Доказать, что $\lambda = 0$ является собственным значением.

\begin{Sol}
Требуемое утверждение эквивалентно $\Ker A \not= \{0\}$, поэтому достаточно найти нетривиальный элемент ядра $A$.

По теореме Гильберта-Шмидта вектор $Ax$ можно разложить в ряд Фурье по собственным векторам $A$ (обозначим их 
$x_1,\ldots x_n$):
$$Ax = \Sum{k=1}{n} \Scal{Ax}{x_k}x_k = \Sum{k=1}{n} \Scal{x}{Ax_k}x_k = \Sum{k=1}{n}\lambda_k \Scal{x}{x_k}x_k$$.
Так как $\Dim \mathbb{H} = \infty$, найдется ненулевой вектор $x$, ортогональный всем собственным векторам 
(их конечное число), откуда $Ax = 0$, что и требовалось.
\end{Sol}
\end{document}


