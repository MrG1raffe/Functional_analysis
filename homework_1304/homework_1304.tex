\documentclass[16pt]{article}
\usepackage[russian]{babel}
\usepackage{a4wide}
\usepackage[utf8]{inputenc}
\usepackage{graphicx}
\usepackage{amsmath}
\usepackage{amssymb}
\usepackage{amsthm}
\usepackage{import}
\usepackage{xifthen}
\usepackage{pdfpages}
\usepackage{transparent}

\newcommand{\incfig}[2]{%
    \def\svgwidth{#2 mm}
    \import{./figures/}{#1.pdf_tex}
}

\newtheorem{Th}{Теорема}
\newenvironment{Proof}{\par\noindent{\bf Доказательство.}}{\hfill$\scriptstyle\blacksquare$}
\newenvironment{Sol}{\par\noindent{\it Решение:}}


\DeclareMathOperator{\Arccos}{arccos}
\DeclareMathOperator{\Arctg}{arctg}
\DeclareMathOperator{\Cl}{Cl}
\DeclareMathOperator{\SI}{SI}
\DeclareMathOperator*{\Argmax}{Argmax}
\DeclareMathOperator*{\Var}{Var}


\newcommand\Real{\mathbb{R}} 
\newcommand\A{(\cdot)} 
\newcommand\Sup[2]{\rho( #1 \, | \, #2 )}
\newcommand\Sum[2]{\sum\limits_{#1}^{#2}}
\newcommand\Scal[2]{\langle #1,\, #2 \rangle}
\newcommand\Norm[1]{\left\| #1 \right\|}
\newcommand\Int[2]{\int\limits_{#1}^{#2}}
\newcommand\PS{\mathcal{P}}
\newcommand\X{\mathcal{X}} 
\newcommand\Pict[3]{
\begin{figure}[h!]
    \centering
    \incfig{#1}{#3}
    \caption{#2}
    \label{fig:#1}
\end{figure}
}



\begin{document}
\paragraph{Домашнее задание от 28-ого дня карантина}
\paragraph{Задача 4.10.}  Найти углы треугольника в $L_2[-1,\,1]$, образованного функциями
 $$x_1(t) = 0, \quad x_2(t) = 1,\quad x_3(t)=t.$$

\begin{Sol}
Вычислим длины сторон.
$$\Norm{x_2-x_1} = \sqrt{2}$$
$$\Norm{x_2 - x_3} = \sqrt{\Int{-1}{1}(1-t)^2dt} = 2\sqrt{\dfrac{2}{3}}$$
$$\Norm{x_1 - x_3} = \sqrt{\Int{-1}{1}t^2dt} = \sqrt{\dfrac{2}{3}}$$
Из соотношений для углов получаем
$$\cos \phi_1 = \dfrac{\Scal{x_2-x_1}{x_3-x_1}}{\Norm{x_2-x_1}\Norm{x_3 - x_1}} = \dfrac{\sqrt{3}}{2}\Int{-1}{1}tdt = 0$$
$$\cos \phi_2 = \dfrac{\Scal{x_1-x_2}{x_3-x_2}}{\Norm{x_1-x_2}\Norm{x_3 - x_2}} = \dfrac{\sqrt{3}}{4}\Int{-1}{1}(1-t)dt 
= \dfrac{\sqrt{3}}{2}$$
$$\cos \phi_3 = \dfrac{\Scal{x_1-x_3}{x_2-x_3}}{\Norm{x_1-x_3}\Norm{x_2 - x_3}} = \dfrac{3}{4}\Int{-1}{1}t(t-1)dt 
= \dfrac{1}{2}$$
Таким образом $\phi_1 = \dfrac{\pi}{2}, \ \phi_2 = \dfrac{\pi}{6}, \ \phi_3 = \dfrac{\pi}{3}$.
\end{Sol}

\paragraph{Задача 4.11.} Для функции $x(t) = t^\alpha$ найти ее норму в пространстве $L_p$ для всех допустимых 
значений $p$.

\begin{Sol}
Значение $p$ допустимо в том и только том случае, когда $\alpha p > -1$.
При $\alpha = 0$ имеем $x(t) = 1, \ \Norm{(x)} = 1$.
Если $\alpha \not= 0$, то
$$\Norm{x}_{L_p} = \left(\Int{0}{1}t^{\alpha p}dt\right)^{\frac{1}{p}} = (\alpha p + 1)^{-\frac{1}{p}}.$$
\end{Sol}

\paragraph{Задача 4.18.}
Пусть $[c,\,d] \subseteq [a,\,b]$. Доказать, что множество
$$M = \{x\A \in L_2[a,\,b]\colon x(t) = 0 \ \text{почти всюду на } [c,\,d]\}$$
является подпространством и описать $M^\bot$.

\begin{Sol}
Очевидно, что $M$ --- линейное многообразие, покажем его замкнутость. Рассмотрим сходящуюся последовательность $x_n \in M, \ x_n \to x$.
$$\Int{c}{d}x^2(t)dt = \Int{c}{d}(x_n(t) - x(t))^2dt \leqslant \Norm{x_n - x}^2 \to 0,$$
поэтому $x(t) = 0$ почти всюду на $[c,\,d]$, то есть $x \in M$. 

Для произвольного элемента $y \in M^\bot$ и для всех $x \in M$ верно
$$\Scal{x}{y} = \Int{a}{b}x(t)y(t)dt = \int\limits_{[a,\,b]\setminus[c,\,d]}x(t)y(t)dt = 0,$$
откуда получаем $y(t) = 0$ почти всюду на $[a,\,b]\setminus[c,\,d]$, и
$$M^\bot = \{x\A \in L_2[a,\,b]\colon x(t) = 0 \ \text{почти всюду на } [a,\,b]\setminus[c,\,d]\}.$$
\end{Sol}

\paragraph{Задача 4.19.}
Доказать, что множество 
$$M = \left\{x\A \in L_2[0,\,1]\colon \Int{0}{1}x(t)dt = 0\right\}$$
является подпространством и описать $M^\bot$.

\begin{Sol}
Функционал $f(x) = \Int{0}{1}x(t)dt$ является непрерывным, поэтому для любой последовательности $x_n \in M,\ x_n \to x$ верно $f(x_n) = 0 = f(x)$, откуда $x \in M$, и множество $M$ является линейным подпространством.
Для $y \in M^\bot, \ x \in M$ выполнено
$$\Scal{x}{y} = \Int{0}{1}x(t)y(t)dt = 0.$$
Это соотношение, очевидно, выполнено при $y(t) = \rm{const}$ почти всюду. Покажем, что для других функций это соотношение
не будет выполнено. Действительно, если функция не эквивалентна постоянной, то найдутся $\gamma_1,\ \gamma_2$ такие, что $\gamma_1 < \gamma_2$ и два множества
$\Delta_1, \Delta_2$ одинаковой меры такие, что $\gamma_1 < \gamma_2$, \ $y(t)|_{\Delta_1} < \gamma_1,\
y(t)|_{\Delta_2} > \gamma_2$. Рассмотрим функцию
$$ x(t) = 
\begin{cases}
1, & t \in \Delta_1\\
-1, & t \in \Delta_2\\
0, & \text{иначе}
\end{cases}
$$
Очевидно, что $x \in M$, однако $\Scal{x}{y} \not= 0$.

Таким образом, $M^\bot$ состоит только из функций, постоянных почти всюду.
\end{Sol}

\paragraph{Задача 5.9.}
В пространстве $L_2[0,\,1]$ найти расстояние от элемента $x(t) = t^2$ до подпространства
$$L = \left\{x\A \in L_2[0,\,1]\colon \Int{0}{1}x(t)dt = 0\right\}.$$

\begin{Sol}
В гильбертовом пространстве $x = y + z$, где $y \in M,\ z \in M^\bot$, причем эти элементы определены 
единственным образом. Из предыдущей задачи следует, что $z(t) = c$ почти всюду, тогда $y(t) = t^2 - c$, при
этом $\Int{0}{1}y(t)dt = 0$, откуда $c = \frac{1}{3}$. Расстояние до подпространства совпадает с длиной перпендикуляра
$$\rho(x, \, L) = \Norm{z} = \frac{1}{3}.$$ 
\end{Sol}

\paragraph{Многочлены Лагерра} Покажем, что полиномы Лагерра, задаваемые выражением
$$x_n(t) = \frac{e^t}{n!}(e^{-t}t^n)^{(n)}$$ 
образуют ортонормированную систему относительно скалярного произведения 
$$\Scal{f}{g} = \Int{0}{\infty}f(t)g(t)e^{-t}dt.$$
Пусть $m < n$.
$$\Scal{t^m}{x_n} = \frac{1}{n!}\Int{0}{\infty}t^m(e^{-t}t^n)^{(n)}dx$$
Интегрируя это выражение $m$ раз по частям, получаем
$$(-1)^m\frac{m!}{n!}\Int{0}{\infty}(e^{-t}t^n)^{(n-m)}dt = 0,$$
поскольку $n - m > 0$. Таким образом, $\Scal{x_n}{x_m} = 0$ для любого $m < n$.   

$$\Scal{x_n}{x_n} = \frac{1}{n!}\Int{0}{\infty}x_n(t)(e^{-t}t^n)^{(n)}dt = \frac{(-1)^n}{n!}\Int{0}{\infty}x_n^{(n)}(t)(e^{-t}t^n)dt.$$
Используя то, что $x_n^{(n)}(t) = (-1)^n$, получаем $\Scal{x_n}{x_n} = 1$.

\paragraph{Полиномы Лежандра}
Покажем, что полиномы 
$$x_n(t) = \frac{1}{2^n n!}\frac{d^n}{dt^n}(t^2-1)^n$$
образуют ортогональную систему в $L_2[-1,\,1]$.
При $m < n$, интегрируя по частям, получаем 
$$\Scal{x_n}{x_m} = \frac{1}{n! m! 2^{n+m}}\Int{-1}{1}\frac{d^n}{dt^n}(t^2-1)^n\frac{d^m}{dt^m}(t^2-1)^mdt = 
\frac{1}{n!m!2^{n+m}}\Int{-1}{1}(t^2-1)^n\frac{d^{m+n}}{dt^{m+n}}(t^2-1)^mdt = 0,$$
поскольку $(t^2-1)^m$ является полиномом степени $2m$.

\paragraph{Полиномы Чебышёва I рода}
Покажем, что система
$$T_n(t) = \cos(n \arccos(t))$$
является ортогональной относительно 
$$\Scal{f}{g} = \Int{-1}{1}f(t)g(t)\frac{dt}{\sqrt{1-t^2}}.$$

Рассмотрим 
$$\Scal{T_n}{T_m} = \Int{-1}{1}T_n(t)T_m(t)\frac{dt}{\sqrt{1-t^2}} = \{t = \cos \phi\} = \Int{0}{\pi}\cos n\phi
\cos m\phi dt = \frac{\pi}{2}\delta_{mn}.$$

\paragraph{Полиномы Чебышёва II рода}
Покажем, что система
$$T_n(t) = \frac{T_{n+1}'(t)}{n+1} = \frac{\sin((n+1)\arccos t)}{\sqrt{1-t^2}}$$
является ортогональной относительно 
$$\Scal{f}{g} = \Int{-1}{1}f(t)g(t)\sqrt{1-t^2}dt.$$
$$\Scal{T_n}{T_m} = \Int{-1}{1}\sin((n+1)\arccos t)\sin((m+1)\arccos t)\frac{dt}{\sqrt{1-t^2}} = \frac{\pi}{2}\delta_{mn}$$
абсолютно аналогично вычислению интеграла для полиномов Чебышёва I рода.
\end{document}


