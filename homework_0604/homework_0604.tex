\documentclass[16pt]{article}
\usepackage[russian]{babel}
\usepackage{a4wide}
\usepackage[utf8]{inputenc}
\usepackage{graphicx}
\usepackage{amsmath}
\usepackage{amssymb}
\usepackage{amsthm}
\usepackage{import}
\usepackage{xifthen}
\usepackage{pdfpages}
\usepackage{transparent}

\newcommand{\incfig}[2]{%
    \def\svgwidth{#2 mm}
    \import{./figures/}{#1.pdf_tex}
}

\newtheorem{Th}{Теорема}
\newenvironment{Proof}{\par\noindent{\bf Доказательство.}}{\hfill$\scriptstyle\blacksquare$}
\newenvironment{Sol}{\par\noindent{\it Решение:}}


\DeclareMathOperator{\Arccos}{arccos}
\DeclareMathOperator{\Arctg}{arctg}
\DeclareMathOperator{\Cl}{Cl}
\DeclareMathOperator{\SI}{SI}
\DeclareMathOperator*{\Argmax}{Argmax}
\DeclareMathOperator*{\Var}{Var}


\newcommand\Real{\mathbb{R}} 
\newcommand\A{(\cdot)} 
\newcommand\Sup[2]{\rho( #1 \, | \, #2 )}
\newcommand\Sum[2]{\sum\limits_{#1}^{#2}}
\newcommand\Scal[2]{\langle #1,\, #2 \rangle}
\newcommand\Norm[1]{\left\| #1 \right\|}
\newcommand\Int[2]{\int\limits_{#1}^{#2}}
\newcommand\PS{\mathcal{P}}
\newcommand\X{\mathcal{X}} 
\newcommand\Pict[3]{
\begin{figure}[h!]
    \centering
    \incfig{#1}{#3}
    \caption{#2}
    \label{fig:#1}
\end{figure}
}



\begin{document}
\paragraph{Домашнее задание от 21-ого дня карантина}
\paragraph{Задача 3.9.} Является ли гильбертовым пространство $\tilde{L}_2[a,b]$ непрерывных функций со
скалярным произведением
$$\Scal{x}{y} = \Int{a}{b}x(t)y(t)dt.$$ 


\begin{Sol}
Без ограничения общности будем рассматривать отрезок $[-1,\,1]$. \\
Последовательность
$$x_n(t) =
\begin{cases}
-1, & t < -\frac{1}{n}\\
nt, & t \in \left[-\frac{1}{n},\,\frac{1}{n}\right]\\
1, & t > \frac{1}{n}
\end{cases}
$$
является фундаментальной относительно нормы, порожденной скалярным произведением, но не сходится по ней,
поэтому пространство не является гильбертовым.
\end{Sol}


\paragraph{Задача 3.11.} Является ли гильбертовым пространство
$$\left\{(x_1, \,x_2,\, \ldots)\colon \sum_{k=1}^\infty x_k^2 < \infty\right\}$$
со скалярным произведением
$$\Scal{x}{y} = \sum_{k=1}^\infty \lambda_k x_k y_k, \quad \lambda_k \in (0,\,1)$$

\begin{Sol}
Не является, поскольку при быстро убывающих $\{\lambda_k\}$ (например, $\Sum{k=1}{\infty} \lambda_k^2 < \infty$)
последовательность 
$$x^{(n)} = (\underbrace{1,\,1,\,\ldots\,1}_n,\,0\,\ldots)$$
является фундаментальной, так как 
$$\Norm{x^{(n+p)}-x^{(n)}} = \Sum{k=n}{n+p}\lambda_k^2 \underset{n\to\infty}{\to} 0$$
в силу критерия Коши сходимости ряда, однако не является сходящейся, поэтому пространство не полно, и следовательно, 
не гильбертово.
\end{Sol}

\paragraph{Задача 3.28.} Показать, что множество 
$$M = \left\{x \in l_2\colon \Sum{k=1}{\infty}x_k = 0\right\}$$
является линейным многообразием, всюду плотным в $l_2$.

\begin{Sol}
Пусть $x, y \in M$. Тогда при любых $\alpha, \beta \in \Real$ верно
$$\Sum{k=1}{\infty} (\alpha x_k + \beta y_k) = \alpha\Sum{k=1}{\infty}x_k + \beta\Sum{k=1}{\infty}y_k = 0,$$
поэтому $M$ является линейным многообразием. Покажем, что $M^\bot = \{0\}$. Из этого по теореме 5 семинара от 
21-го дня карантина будет следовать, что $M$ плотно в $l_2$.

Пусть $y \in M^\bot$. Тогда для любого $x \in M$ 
$$\Scal{x}{y} = 0.$$
Рассматривая $$x^{(n)} = (\underbrace{0,\,\ldots,0}_{n-1},\,1,-1,0,\ldots), \quad n \in \mathbb{N},$$
получим, что $y_n = y_{n+1}$ для любого $n \in \mathbb{N}$. Однако $y \in l_2$, и потому 
$\Sum{k=1}{\infty}y_k^2 < \infty$, откуда следует, что $y_n = 0$ при любом $n \in \mathbb{N}$, а значит $y = 0$,
и $M^\bot = \{0\}$.
\end{Sol}

\paragraph{Задача 3.36.} В пространстве $\tilde{L}_2[-1,\,1]$ задано множество
$$M = \{x\A \in \tilde{L}_2[-1,\,1]\colon x(t) = 0,\ t \geqslant 0\}.$$
Показать, что $M$ является линейным подпространством $\tilde{L}_2[-1,\,1]$, описать $M^\bot$ и проверить, выполнено ли
\begin{equation} \label{eq}
\tilde{L}_2[-1,\,1] = M \oplus M^\bot
\end{equation}

\begin{Sol}
Очевидно, что $M$ является линейным многообразием, поскольку $\alpha x(t) + \beta y(t) = 0$ при $t \geqslant 0$.
Покажем, что оно замкнуто. Пусть $x_n\A \in M, \quad x_n \overset{\Norm{\cdot}}{\to} x$. Тогда
$$\Int{0}{1}x^2(t)dt = \Int{0}{1}(x_n(t) - x(t))^2dt \leqslant \Int{-1}{1}(x_n(t) - x(t))^2dt \to 0,$$
откуда следует, что $x(t) = 0$ для почти всех $t \in [0,\,1]$. Но так как функция $x\A$ непрерывна, то 
$x(t) = 0, \ t \in [0,\,1]$, то есть $x\A \in M$. Таким образом $M$ является замкнутым линейным многообразием, то
есть линейным  подпространством $\tilde{L}_2[-1,\,1]$.

Пусть теперь $y\A \in M^\bot$. Тогда для любого $x\A \in M$
$$\Scal{x}{y} = \Int{-1}{1}x(t)y(t)dt = \Int{-1}{0}x(t)y(t)dt = 0,$$
поэтому $y(t) = 0$ почти всюду на $[-1,\,0]$, а значит $y(t) = 0,\ t \in [-1,\,0]$.
Итак, 
$$M^\bot = \{y\A \in \tilde{L}_2[-1,\,1]\colon y(t) = 0,\ t \leqslant 0\}.$$

Однако (\ref{eq}) не выполнено, поскольку функцию $z(t) \equiv 1$ нельзя представить в виде 
$$z\A = x\A + y\A. \quad x\A\in M,\, y\A \in M^\bot,$$
так как $x(0) = y(0) = 0$ для любых $x\A \in M,\ y\A \in M^\bot$, однако $z(0) = 1$.
\end{Sol}

\paragraph{Задача 3.37.} Пусть $x_k \in l_2$, и
$$x_k = \left(1,\, \frac{1}{2^k},\,\frac{1}{2^{2k}},\,\frac{1}{2^{3k}},\,\ldots\right).$$
Доказать, что линейная оболочка множества $M = \{x_k, \ k \in \mathbb{N}\}$ плотна в $l_2$.

\begin{Sol}
Покажем, что $M^\bot = \{0\}$. Для произвольного $y \in M^\bot$ и любого $k$ верно
$$\Scal{y}{x_k} = y_1 + \frac{1}{2^k}\Sum{n=2}{\infty}\frac{y_n}{2^{(n-2)k}} = 0.$$
Поскольку $\Sum{n=2}{\infty}\dfrac{y_n}{2^{(n-2)k}} < \infty$, устремляя $k$ к бесконечности, получим $y_1 =0$.

Далее,
$$2^k\Scal{y}{x_k} = y_2 + \frac{1}{2^k}\Sum{n=3}{\infty}\frac{y_n}{2^{(n-3)k}} = 0.$$
Вновь переходя к пределу при $k \to \infty $, получаем $y_2 = 0$.

Продолжая так, получим $y = 0$. Значит, $M^\bot = \{0\}$, что и требовалось доказать.
\end{Sol}

\paragraph{Задача 3.44.} Доказать, что в гильбертовом пространстве $\mathbb{H}$ любая последовательность непустых
вложенных выпуклых замкнутых ограниченных множеств $A_n$ имеет непустое пересечение.

\begin{Sol}
Выберем последовательность $x_n \in A_n$ таких, что $\rho(0, x_n) = \rho(0, A_n)$, то есть наименьший по норме 
элемент $A_n$. Это можно сделать для замкнутого выпулого множества по теореме 1 семинара от 21-го дня карантина.
Заметим, что последовательность $d_n = \Norm{x_n}$ возрастает в силу вложенности множеств $A_n$ и ограничена 
сверху в силу
ограниченности множества $A_1$, а значит имеет предел, равный $d$. Покажем, что последовательность $x_n$
 фундаментальна: из тождества параллелограмма следует, что для любых $n, \, p \in \mathbb{N}$
$$\Norm{x_n + x_{n+p}}^2 + \Norm{x_n - x_{n+p}}^2 = 2\left(\Norm{x_n}^2 + \Norm{x_{n+p}}^2\right).$$

Заметим, что $\dfrac{x_n+x_{n+p}}{2}$ лежит в $A_n$ в силу выпуклости множества, поэтому 
$2\Norm{\dfrac{x_n+x_{n+p}}{2}} \geqslant~2d_n,$ и значит
$$ \Norm{x_n - x_{n+p}}^2 \leqslant 2\left(\Norm{x_n}^2 + \Norm{x_{n+p}}^2\right) - (2d_n)^2 \to 0.$$
Получили, что последовательность $\{x_n\}$ фундаментальна, и следовательно, сходится к элементу $x_0$
($\mathbb{H}$ гильбертово), который по построению принадлежит $\bigcap\limits_{n=1}^\infty A_n$.
\end{Sol}
 
 \paragraph{Задача 3.45.} Построить в $C[0,\,1]$ пример последовательности непустых
вложенных выпуклых замкнутых ограниченных множеств $A_n$, имеющих пустое пересечение.

\begin{Sol}
Рассмотрим последовательность 
$$A_n = \{x\A \in C[0,\,1]\colon 0 \leqslant x(t) \leqslant t^n, \ x(1) = 1\}.$$

Очевидно, что она удовлетворяет всем описанным условиям, однако $\bigcap\limits_{n=1}^\infty A_n= \varnothing$.
\end{Sol}
\end{document}

