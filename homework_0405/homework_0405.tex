\documentclass[16pt]{article}
\usepackage[russian]{babel}
\usepackage{a4wide}
\usepackage[utf8]{inputenc}
\usepackage{graphicx}
\usepackage{amsmath}
\usepackage{mathrsfs}  
\usepackage{amssymb}
\usepackage{amsthm}
\usepackage{import}
\usepackage{xifthen}
\usepackage{pdfpages}
\usepackage{transparent}

\newcommand{\incfig}[2]{%
    \def\svgwidth{#2 mm}
    \import{./figures/}{#1.pdf_tex}
}

\newtheorem{Th}{Теорема}
\newenvironment{Proof}{\par\noindent{\bf Доказательство.}}{\hfill$\scriptstyle\blacksquare$}
\newenvironment{Sol}{\par\noindent{\it Решение:}}


\DeclareMathOperator{\Arccos}{arccos}
\DeclareMathOperator{\Arctg}{arctg}
\DeclareMathOperator{\Cl}{Cl}
\DeclareMathOperator{\Ker}{ker}
\DeclareMathOperator{\Ima}{im}
\DeclareMathOperator*{\Argmax}{Argmax}
\DeclareMathOperator*{\Var}{Var}


\newcommand\Real{\mathbb{R}} 
\newcommand\A{(\cdot)} 
\newcommand\Sup[2]{\rho( #1 \, | \, #2 )}
\newcommand\Sum[2]{\sum\limits_{#1}^{#2}}
\newcommand\Scal[2]{\langle #1,\, #2 \rangle}
\newcommand\Norm[1]{\left\| #1 \right\|}
\newcommand\Int[2]{\int\limits_{#1}^{#2}}
\newcommand\PS{\mathcal{P}}
\newcommand\X{\mathcal{X}} 
\newcommand\Pict[3]{
\begin{figure}[h!]
    \centering
    \incfig{#1}{#3}
    \caption{#2}
    \label{fig:#1}
\end{figure}
}



\begin{document}
\paragraph{Домашнее задание от 49-ого дня карантина}
\paragraph{Задача 19.4.(б)} Найти собственные значения и собственные векторы оператора
$$Ax(t) = \Int{-\pi}{\pi}\cos(s+t)x(s)ds,$$
действующего в пространстве $C[-\pi,\,\pi]$.
\begin{Sol}
$$Ax(t) = \cos(t)\Int{-\pi}{\pi}cos(s)x(s)ds - \sin(t)\Int{-\pi}{\pi}sin(s)x(s)ds = \lambda x(t)$$

Если $\lambda = 0$, то собственными векторами являются все элементы, ортогональные к $\sin(t)$ и $\cos(t)$ в
смысле скалярного произведения в $L_2$. 

При $\lambda \not= 0$ нужно искать решение в виде линейной комбинации $x(t) = c_1 \sin(t) + c_2 \cos(t)$
Приравнивая коэффициенты в обеих частях уравнения, получаем
\[
\begin{cases}
	c_1 \pi = -\lambda c_1 \\
	c_2 \pi = \lambda c_2
\end{cases}
\] 
Здесь возможны всего два случая: 
\begin{enumerate}
\item $c_1 = 0$, тогда $\lambda = \pi$, и собственный вектор $x(t) = \cos(t)$.
\item $c_2 = 0,\ \lambda = -\pi$, собственный вектор $x(t) = \sin(t)$.
\end{enumerate}
\end{Sol} 

\paragraph{Задача 19.5.} Найти собственные значения и собственные векторы оператора $Ax(t) = x''(t)$, действующего
в пространстве $C[0,\,\pi]$, если область определения имеет вид
\begin{enumerate}
\item $D(A) = \{x\A \in C[0,\,\pi]\colon x''\A \in C[0,\,\pi],\ x(0) = x(\pi) = 0\}$
\item $D(A) = \{x\A \in C[0,\,\pi]\colon x''\A \in C[0,\,\pi],\ x'(0) = x'(\pi) = 0\}$
\item $D(A) = \{x\A \in C[0,\,\pi]\colon x''\A \in C[0,\,\pi],\ x(0) = x(\pi),\ x'(0) = x'(\pi)\}$
\end{enumerate}

\begin{Sol}
Если $\lambda > 0$, то решением уравнения
$$x''(t) = \lambda x(t)$$
является функция $x(t) = e^{\sqrt{\lambda}t}$, которая не удовлетворяет краевым условиям, поэтому будем
рассматривать $\lambda \leqslant 0$. В этом случае решениями являются функции $\cos(\sqrt{\lambda}t), \  \sin(\sqrt{\lambda}t)$. Из краевых условий получаем
\begin{enumerate}
\item $\lambda = -k^2,\ x(t) = \sin(k\lambda),\ k = 1, 2, \ldots$ 
\item $\lambda = -k^2,\ x(t) = \cos(k\lambda),\ k = 0, 1, 2, \ldots$ 
\item $\lambda = -4k^2,\ x(t) = \sin(2k\lambda), \ x(t) = \cos(2k\lambda), \ k = 1, 2, \ldots$ 
\end{enumerate}
\end{Sol}

\paragraph{Задача 19.13.} Пусть $A \in \mathscr{L}(X)$. Может ли $R_\lambda(A) = (A - \lambda I)^{-1}$ быть вполне
непрерывным?
\begin{Sol}
Резольвента является непрерывным оператором, поэтому в конечномерных пространствах, где непрерывность эквивалентна
полной непрерывности, утверждение выполнено.

В случае бесконечномерного пространства заметим, что, в силу принципа Банаха об открытости, образ открытого единичного
шара при действии непрерывного оператора открыт, а значит, содержит в себе шар некоторого радиуса, не являющийся
предкомпактом в бесконечномерном банаховом пространстве. Образ замкнутого единичного шара тогда содержит в себе
непредкомпактное множество, и потому сам не является предкомпактным. 

Таким образом, $R_\lambda$ вполне непрерывен
в том и только том случае, когда пространство конечномерно.
\end{Sol}

\paragraph{Задача 19.14.} Рассмотрим в $C[0,\,1]$ оператор $Ax(t) = tx(t)$. Доказать, что $\sigma(A) = [0,\,1]$
и ни одна из точек спектра не является собственным значением.
\begin{Sol}
Так как $\Norm{A} = 1$, получаем $\sigma(A) \subset [-1,\,1]$.
Уравнение $(A - \lambda I)x = y$ принимает вид:
$$(t-\lambda)x(t) = y(t),$$
отсюда при $\lambda < 0$ оператор непрерывно обратим, и
$$(A - \lambda I)^{-1}y(t) = \frac{y(t)}{t - \lambda}.$$
При $\lambda \geqslant 0$ этот оператор разрывен, и потому
$\sigma(A) = [0,\,1]$.

В спектре отсутствуют собственные значения, поскольку уравнение
$$tx(t) = \lambda x(t)$$
имеет лишь нулевое решение в классе непрерывных функций.
\end{Sol}

\paragraph{Задача 19.18.} В $C[0,\,1]$ задан оператор $Ax(t) = \Int{0}{t} x(\tau)d\tau$. Найти $\sigma(A), \ 
R_\lambda(A)$.

\begin{Sol}
Так как $\Norm{A} = 1$, $\ \sigma(A) \subset [-1,\,1]$.
Уравнение $(A-\lambda I)x = y$ эквивалентно
$$\Int{0}{t}x(\tau)d\tau = \lambda x(t) + y(t)$$
Так как левая часть непрерывно дифференцируема, обозначая $z(t) = \lambda x(t) + y(t)$ и дифференцируя обе
части, находим
$$x(t) = \dot{z}(t).$$
При $\lambda \not= 0$ получаем линейное дифференциальное уравнение
$$\dot{z}(t) + \frac{z(t)}{\lambda} - \frac{y(t)}{\lambda} = 0,$$
решение которого находится по формуле Коши. Учитывая $z(0) = 0$, получаем
$$z(t) = \frac{1}{\lambda}\Int{0}{t}e^{\frac{1}{\lambda}(\tau - t)}y(\tau)d\tau,$$ 
откуда
$$x(t) = \frac{1}{\lambda^2}\Int{0}{t}e^{\frac{1}{\lambda}(\tau - t)}y(\tau)d\tau - \frac{1}{\lambda}y(t).$$ 
Очевидно, что этот оператор непрерывен, и потому $[-1,\,1]\setminus \{0\} \subset \rho(A)$.
Но так как спектр непрерывного оператора непуст, то он состоит из единственной точки $\sigma(A) = \{0\}$.
\end{Sol}
\end{document}


