\documentclass[16pt]{article}
\usepackage[russian]{babel}
\usepackage{a4wide}
\usepackage[utf8]{inputenc}
\usepackage{graphicx}
\usepackage{amsmath}
\usepackage{amssymb}
\usepackage{amsthm}
\usepackage{import}
\usepackage{xifthen}
\usepackage{pdfpages}
\usepackage{transparent}

\newcommand{\incfig}[2]{%
    \def\svgwidth{#2 mm}
    \import{./figures/}{#1.pdf_tex}
}

\newtheorem{Th}{Теорема}
\newenvironment{Proof}{\par\noindent{\bf Доказательство.}}{\hfill$\scriptstyle\blacksquare$}
\newenvironment{Sol}{\par\noindent{\it Решение:}}


\DeclareMathOperator{\Arccos}{arccos}
\DeclareMathOperator{\Arctg}{arctg}
\DeclareMathOperator{\Cl}{Cl}
\DeclareMathOperator{\Ker}{ker}
\DeclareMathOperator{\Ima}{im}
\DeclareMathOperator*{\Argmax}{Argmax}
\DeclareMathOperator*{\Var}{Var}


\newcommand\Real{\mathbb{R}} 
\newcommand\A{(\cdot)} 
\newcommand\Sup[2]{\rho( #1 \, | \, #2 )}
\newcommand\Sum[2]{\sum\limits_{#1}^{#2}}
\newcommand\Scal[2]{\langle #1,\, #2 \rangle}
\newcommand\Norm[1]{\left\| #1 \right\|}
\newcommand\Int[2]{\int\limits_{#1}^{#2}}
\newcommand\PS{\mathcal{P}}
\newcommand\X{\mathcal{X}} 
\newcommand\Pict[3]{
\begin{figure}[h!]
    \centering
    \incfig{#1}{#3}
    \caption{#2}
    \label{fig:#1}
\end{figure}
}



\begin{document}
\paragraph{Домашнее задание от 42-ого дня карантина}
\paragraph{Задача 16.1.} Являются ли вполне непрерывными следующие операторы $$A\colon C[0,\,1] \to C[0,\,1]?$$
\begin{enumerate}
\item $(Ax)(t) = x(0) + tx(1)$
\item $(Ax)(t) = \Int{0}{1}e^{ts}x(s)ds$
\item $(Ax)(t) = x(t^2)$
\end{enumerate}
\begin{Sol}
\begin{enumerate}
\item Для образа единичного шара
$|(Ax)(t)| \leqslant 2\Norm{x}_C \leqslant 2,\ $
$|(Ax)(t+h)-(Ax)(t)| = h|x(1)| \leqslant h.$
Из теоремы Арцела-Асколи следует полная непрерывность $A$.
\item $|(Ax)(t)| \leqslant e\Norm{x} \leqslant e, \ |(Ax)(t+h)-(Ax)(t)| = \left|\Int{0}{1}e^{ts}(e^{hs}-1)x(s)ds\right|
\leqslant e(e^h - 1)$, то есть выполнены условия теоремы Арцела-Асколи.
\item
Пусть $x_n = t^n$. $\Norm{x_n} \leqslant 1$, при этом из $Ax_n = t^{2n}$ нельзя выделить фундаментальную 
подпоследовательность, следовательно, $A$ не является вполне непрерывным.
\end{enumerate}
\end{Sol}

\paragraph{Задача 16.5.} Будет ли оператор $(Ax)(t) = x'(t)$ вполне непрерывным, если он действует 
\begin{enumerate}
\item $A\colon C^1[0,\,1] \to C[0,\,1]$
\item $A\colon C^2[0,\,1] \to C^1[0,\,1]$
\item $A\colon C^2[0,\,1] \to C[0,\,1]$
\end{enumerate}
\begin{Sol}
\begin{enumerate}
\item Нет, достаточно рассмотреть последовательность $x_n = \frac{t^n}{n}$.
\item Нет, рассмотрим последовательность $x_n = \frac{t^{n+1}}{n(n+1)}$. Она ограничена в $C^2[0,\,1]$,
но ее образ не предкомпактен, так как из $(Ax_n)' = t^{n-1}$ нельзя выделить фундаментальную в 
$C[0,\,1]$ подпоследовательность,
а значит, нельзя выделить фундаментальную в $C^1[0,\,1]$ подпоследовательность из $\{Ax_n\}$.
\item Для $x$ из единичного шара в $C^2[0,\,1]$ верно
$$x'(t) \leqslant \Norm{x}_{C^2} \leqslant 1,$$
$$|x'(t+h) - x'(t)| \leqslant |x''(\xi)|h \leqslant h.$$
Пользуясь теоремой Арцела-Асколи, получаем предкомпактность образа единичного шара, а следовательно, полную
непрерывность $A$.
\end{enumerate}
\end{Sol}

\paragraph{Задача 16.9.} Доказать, что оператор вложения $Ix = x$, действующий из
\begin{enumerate}
\item $I\colon C^1[a,\,b] \to C[a,\,b]$
\item $I\colon H^1[a,\,b] \to C[a,\,b]$
\end{enumerate}
вполне непрерывен.

\begin{Sol}
\begin{enumerate}
\item $\Norm{x}_C \leqslant \Norm{x}_{C^1},\ |x(t+h) - x(t)| = |x'(\xi)|h \leqslant h$. Из теоремы Арцела-Асколи
получаем требуемое утверждение.
\item Вспомним эквивалентное определение вполне непрерывного оператора: оператор $A$ вполне непрерывен, если переводит
слабо сходящуюся последовательность в сильно сходящуюся. Пусть $x_n\A$ сходится к $x\A$ в $H^1$ слабо. Так как $H^1$ 
гильбертово, это означает, что для любой $f\A \in H^1$
$$\Int{a}{b}[f(t)(x_n(t) - x(t)) + f'(t)(x_n'(t)-x'(t))]dt \to 0.$$
Предположим, что $\Norm{x_n-x}_C \nrightarrow 0$. Тогда, в силу непрерывности функций, найдется интервал, на котором
$\lim\sup|x_n(t)-x(t)| > \delta$ для некоторого $\delta > 0$. Взяв в качестве $f$ индикатор этого интервала
(его можно приблизить непрерывно дифференцируемыми функциями), получаем противоречие.
Значит, любая слабо сходящаяся в $H^1$ последовательность является сильно сходящейся в $C$, и оператор $A$ вполне 
непрерывен.
\end{enumerate}
\end{Sol}

\paragraph{Задача 16.12.} Оператор $Ax = x''$ определен на множестве дважды непрерывно дифференцируемых функций
с $x(0) = x(1) = 0$. Доказать существование $A^{-1}$, найти его и доказать, что он является вполне непрерывным.
\begin{Sol}
Обратный оператор определен на множестве непрерывных функций. Решая уравнение $x'' = y$ с краевыми условиями 
$x(0) = x(1) = 0$, находим 
$$x(t) = \Int{0}{t}\Int{0}{\tau}y(s)dsd\tau - t\Int{0}{1}\Int{0}{\tau}y(s)dsd\tau.$$
Для непрерывных $y\A$ из единичного шара относительно нормы $\Norm{\cdot}_{L_2}$ 
$$\Norm{x\A}_{L_2} \leqslant 4,$$
поскольку $\max |x(t)| \leqslant 2$ (неравенство треугольника и неравенство Гёльдера). Значит, множество $\{A^{-1}(y)\}$ равномерно
ограничено.
$$\Int{0}{1-h}[x(t+h)-x(t)]^2dt = \Int{0}{1-h}\left(\Int{t}{t+h}\Int{0}{\tau}y(s)dsd\tau - 
h\Int{0}{1}\Int{0}{\tau}y(s)dsd\tau\right)^2dt \leqslant$$
$$ \leqslant \Int{0}{1-h}\left(h+h\left|\Int{0}{1}\Int{0}{\tau}y(s)dsd\tau\right|
\right)^2dt \leqslant \leqslant 4h^2.$$
Из критерия предкомпактности в $L_2$ получаем требуемое.
\end{Sol}

\paragraph{Задача 16.20.} Пусть $\{e_n\}$ --- ортонормированный базис в гильбертовом пространстве $\mathbb{H}$.
$\lambda_n \in \Real, \ \lambda_n \to 0$. Пусть
$$Ax = \Sum{n=1}{\infty}\lambda_n\Scal{x}{e_n}e_n.$$
Показать, что $A$ определен на всем $\mathbb{H}$, переводит его в себя и является вполне непрерывным.
\begin{Sol}
Гильбертово пространство с счётным базисом изоморфно $l_2$, поэтому будем отождествлять элементы $x \in \mathbb{H}$ 
с $x \in l_2$.

Тогда видно, что оператор
$$Ax = (\lambda_1x_1,\, \lambda_2x_2,\, \ldots)$$
определен для всех $x \in l_2$ и что $Ax \in l_2$ поскольку $\{\lambda_n\}$ ограничена и не влияет на сходимость ряда.

Покажем, что образ единичного шара при действии $A$ является предкомпактным.
$$\Norm{Ax} \leqslant \max|\lambda_n|\Norm{x},$$
поэтому множество равномерно ограничено.
$$\Sum{n=N}{\infty}\lambda_n^2x_n^2 \leqslant \max_{n \geqslant N}|\lambda_n| \to 0.$$
Из критерия предкомпактности в $l_2$ получаем полную непрерывность $A$.
\end{Sol}

\paragraph{Задача 16.26.} Доказать, что множество значений вполне непрерывного оператора $A$ сепарабельно.
\begin{Sol}
Покажем, что образ единичного шара имеет счетное всюду плотное множество.
Действительно, он является предкомпактным, а значит, вполне ограничен, и для любого $\epsilon > 0$ можно найти
конечную $\epsilon-$сеть. Рассмотрим $\epsilon_n = \frac{1}{n}$ и возьмем $S$ --- объединение всех $\epsilon_n$-сетей.
Это множество и будет искомым счетным всюду плотным множеством.
Множество значений $A$ можно представить как $\bigcup\limits_rA(B_r(0))$, для $A(B_r(0))$ множество $rS$ является
счетным всюду плотным, поэтому $\bigcup\limits_{r \in \mathbb{Q}}rS$ будет счетным всюду плотным множеством 
для множества значений $A$.
\end{Sol}
\end{document}


